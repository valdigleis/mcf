%---------------------------------------------------------------------------------------
%	Define a caixinha para o ambiente  de definição
%---------------------------------------------------------------------------------------

% nome do estilo para usar nas definições
\newtheoremstyle{defstyle}
% Espaço acima
{0pt}
% Espaço abaixo
{0pt}
% Fonte dentro da caixa
{\normalfont}
% Indentação
{}
% Título
{\small\bf\sffamily}
% Pontuação pós título
{\;}
% Espaço depois do título
{0.25em}
% Formatação do texto do título para a caixinha de Teorema
{
  \small\sffamily\thmname{#1}\nobreakspace\thmnumber{\@ifnotempty{#1}{}\@upn{#2}}
  \thmnote{\nobreakspace\the\thm@notefont\sffamily\bfseries---\nobreakspace#3.}
}

% Define a forma de numeração as definições
\theoremstyle{defstyle}
\newtheorem{defT}{Definição}[chapter]

% Define a caixa para as definições
\newmdenv[
  skipabove=7pt,
  skipbelow=7pt,
  rightline=true,
  leftline=true,
  topline=true,
  bottomline=true,
  linecolor=NordFrost1,
  backgroundcolor=white,
  innerleftmargin=5pt,
  innerrightmargin=5pt,
  innertopmargin=5pt,
  innerbottommargin=5pt,
  leftmargin=0cm,
  rightmargin=0cm,
  linewidth=1pt]{defBox}

% Cria o ambiente para a definição
\newenvironment{definicao}{\begin{defBox}\begin{defT}}{\end{defT}\end{defBox}}



%---------------------------------------------------------------------------------------
%	Define a caixinha para o ambiente  de Notas
%---------------------------------------------------------------------------------------

% nome do estilo para usar nas notas
\newtheoremstyle{notestyle}
% Espaço acima
{0pt}
% Espaço abaixo
{0pt}
% Fonte dentro da caixa
{\normalfont}
% Indentação
{}
% Título
{\small\bf\sffamily}
% Pontuação pós título
{\;}
% Espaço depois do título
{0.2em}
% Formatação do texto do título para a caixinha de notação
{
  \small\sffamily\thmname{#1}\nobreakspace\thmnumber{\@ifnotempty{#1}{}\@upn{#2}}
  \thmnote{\nobreakspace\the\thm@notefont\sffamily\bfseries--\nobreakspace#3}
}

% Define a forma de numeração as observações de notação
\theoremstyle{notestyle}
\newtheorem{noteT}{\color{NordAurora2}Tomando Notas}[chapter]

% Define a caixa para as definições
\newmdenv[
  skipabove=7pt,
  skipbelow=7pt,
  rightline=true,
  leftline=true,
  topline=true,
  bottomline=true,
  linecolor=NordAurora2,
  backgroundcolor=white,
  innerleftmargin=5pt,
  innerrightmargin=5pt,
  innertopmargin=5pt,
  innerbottommargin=5pt,
  leftmargin=0cm,
  rightmargin=0cm,
  linewidth=1pt]{noteBox}

% Cria o ambiente para notas (ALiCIA escrevendo)
\newenvironment{nota}{
  \begin{noteBox}
    \printALiCIAWritte
    \begin{noteT}
}{
    \end{noteT}
  \end{noteBox}
}



%---------------------------------------------------------------------------------------
%	Define a caixinha para o ambiente de exemplo
%---------------------------------------------------------------------------------------

% nome do estilo para usar nas notas
\newtheoremstyle{examplestyle}
% Espaço acima
{0pt}
% Espaço abaixo
{0pt}
% Fonte dentro da caixa
{\normalfont}
% Indentação
{}
% Título
{\small\bf\sffamily}
% Pontuação pós título
{\;}
% Espaço depois do título
{0.25em}
% Formatação do texto do título para a caixinha de notação
{
  \small\bf\normalfont\thmname{#1}\nobreakspace\thmnumber{\@ifnotempty{#1}{}\@upn{#2}}
  \thmnote{\nobreakspace\the\thm@notefont\sffamily\bfseries---\nobreakspace#3}
}

% Define a forma de numeração dos exemplos
\theoremstyle{examplestyle}
\newtheorem{exampleT}{\color{NordAurora3}Exemplo}[chapter]

% Define a caixa para os exemplos
\newmdenv[
  skipabove=8pt,
  skipbelow=8pt,
  rightline=false,
  leftline=false,
  topline=true,
  bottomline=true,
  linecolor=NordAurora3,
  backgroundcolor=white,
  innerleftmargin=5pt,
  innerrightmargin=5pt,
  innertopmargin=5pt,
  innerbottommargin=5pt,
  leftmargin=0cm,
  rightmargin=0cm,
  linewidth=0.5pt]{exampleBox}

% Cria o ambiente para exemplos
\newenvironment{exemplo}{\begin{exampleBox}\begin{exampleT}}{\end{exampleT}\end{exampleBox}}



%---------------------------------------------------------------------------------------
%	Define a caixinha para o ambiente dicas e cuidado
%---------------------------------------------------------------------------------------

% nome do estilo para usar nas notas
\newtheoremstyle{observacaostyle}
% Espaço acima
{2pt}
% Espaço abaixo
{2pt}
% Fonte dentro da caixa
{\normalfont}
% Indentação
{}
% Título
{\small\bf\sffamily}
% Pontuação pós título
{\;}
% Espaço depois do título
{0.2em}
% Formatação do texto do título para a caixinha de notas
{
  \small\sffamily\thmname{#1}\nobreakspace\thmnumber{\@ifnotempty{#1}{}\@upn{#2}}
  \thmnote{\nobreakspace\the\thm@notefont\sffamily\bfseries---\nobreakspace#3}
}

% Define a forma de numerar as observações de notação
\theoremstyle{observacaostyle}
\newtheorem{observacaoT}{\color{NordAurora4}Observação}[chapter]

% Define a caixa para as observações
\newmdenv[
  skipabove=7pt,
  skipbelow=7pt,
  rightline=true,
  leftline=true,
  topline=true,
  bottomline=true,
  linecolor=NordAurora4,
  backgroundcolor=white,
  innerleftmargin=5pt,
  innerrightmargin=5pt,
  innertopmargin=5pt,
  innerbottommargin=5pt,
  leftmargin=0cm,
  rightmargin=0cm,
  linewidth=1pt]{observacaoBox}

% Cria o ambiente dicas (ALiCIA Confusa)
\newenvironment{dica}{
  \begin{observacaoBox}
    \printALiCIAConfused
    \begin{observacaoT}
}{
    \end{observacaoT}
  \end{observacaoBox}
}

% Cria o ambiente cuidado  (ALiCIA Confusa)
\newenvironment{cuidado}{
  \begin{observacaoBox}
    \printALiCIARage
    \begin{observacaoT}
}{
    \end{observacaoT}
  \end{observacaoBox}
}


%---------------------------------------------------------------------------------------
%	Define a caixinha para o ambiente  de teorema
%---------------------------------------------------------------------------------------

% nome do estilo para usar nos teoremas
\newtheoremstyle{teostyle}
% Espaço acima
{0pt}
% Espaço abaixo
{0pt}
% Fonte dentro da caixa
{\normalfont}
% Indentação
{}
% Título
{\small\bf\sffamily}
% Pontuação pós título
{\;}
% Espaço depois do título
{0.25em}
% Formatação do texto do título para a caixinha de Teorema
{
  \small\sffamily\thmname{#1}\nobreakspace\thmnumber{\@ifnotempty{#1}{}\@upn{#2}}
  \thmnote{\nobreakspace\the\thm@notefont\sffamily\bfseries---\nobreakspace#3.}
}

% Define a forma de numerar os teoremas
\theoremstyle{teostyle}
\newtheorem{teoT}{Teorema}[chapter]

% Define a caixa para as teoremas
\newmdenv[
  skipabove=7pt,
  skipbelow=7pt,
  rightline=true,
  leftline=true,
  topline=true,
  bottomline=true,
  linecolor=NordPolarNight1,
  backgroundcolor=white,
  innerleftmargin=5pt,
  innerrightmargin=5pt,
  innertopmargin=5pt,
  innerbottommargin=5pt,
  leftmargin=0cm,
  rightmargin=0cm,
  linewidth=1pt]{teoBox}

% Cria o ambiente para o teorema
\newenvironment{teorema}{\begin{teoBox}\begin{teoT}}{\end{teoT}\end{teoBox}}

%---------------------------------------------------------------------------------------
%	Define a caixinha para o ambiente  de corolario
%---------------------------------------------------------------------------------------

% nome do estilo para usar nos corolários
\newtheoremstyle{colstyle}
% Espaço acima
{0pt}
% Espaço abaixo
{0pt}
% Fonte dentro da caixa
{\normalfont}
% Indentação
{}
% Título
{\small\bf\sffamily}
% Pontuação pós título
{\;}
% Espaço depois do título
{0.25em}
% Formatação do texto do título para a caixinha de Corolário
{
  \small\sffamily\thmname{#1}\nobreakspace\thmnumber{\@ifnotempty{#1}{}\@upn{#2}}
  \thmnote{\nobreakspace\the\thm@notefont\sffamily\bfseries---\nobreakspace#3.}
}

% Define a forma de numerar os corolários
\theoremstyle{colstyle}
\newtheorem{colT}{\color{NordAurora5}Corolário}[chapter]

% Define a caixa para os corolários
\newmdenv[
  skipabove=5pt,
  skipbelow=5pt,
  rightline=false,
  leftline=false,
  topline=false,
  bottomline=false,
  linecolor=NordPolarNight1,
  backgroundcolor=white,
  innerleftmargin=0pt,
  innerrightmargin=0pt,
  innertopmargin=3pt,
  innerbottommargin=3pt,
  leftmargin=0cm,
  rightmargin=0cm,
  linewidth=1pt]{colBox}

% Cria o ambiente para o corolário
\newenvironment{corolario}{\begin{colBox}\begin{colT}}{\end{colT}\end{colBox}}

%---------------------------------------------------------------------------------------
%	Define a caixinha para o ambiente  de proposição
%---------------------------------------------------------------------------------------

  % nome do estilo para usar nas proposições
\newtheoremstyle{propstyle}
% Espaço acima
{0pt}
% Espaço abaixo
{0pt}
% Fonte dentro da caixa
{\normalfont}
% Indentação
{}
% Título
{\small\bf\sffamily}
% Pontuação pós título
{\;}
% Espaço depois do título
{0.25em}
% Formatação do texto do título para a caixinha de proposição
{
  \small\sffamily\thmname{#1}\nobreakspace\thmnumber{\@ifnotempty{#1}{}\@upn{#2}}
  \thmnote{\nobreakspace\the\thm@notefont\sffamily\bfseries---\nobreakspace#3.}
}

% Define a forma de numerar as proposições
\theoremstyle{propstyle}
\newtheorem{propT}{Proposição}[chapter]

% Define a caixa para as proposições
\newmdenv[
  skipabove=5pt,
  skipbelow=5pt,
  rightline=false,
  leftline=true,
  topline=false,
  bottomline=false,
  linecolor=NordAurora4,
  backgroundcolor=white,
  innerleftmargin=0pt,
  innerrightmargin=0pt,
  innertopmargin=3pt,
  innerbottommargin=3pt,
  leftmargin=0cm,
  rightmargin=0cm,
  linewidth=3pt]{propBox}

% Cria o ambiente para a proposição
\newenvironment{proposicao}{\begin{propBox}\begin{propT}}{\end{propT}\end{propBox}}

%---------------------------------------------------------------------------------------
%	Define a caixinha para o ambiente  de princípio
%---------------------------------------------------------------------------------------

% nome do estilo para usar nos princípio
\newtheoremstyle{prinstyle}
% Espaço acima
{0pt}
% Espaço abaixo
{0pt}
% Fonte dentro da caixa
{\normalfont}
% Indentação
{}
% Título
{\small\bf\sffamily}
% Pontuação pós título
{\;}
% Espaço depois do título
{0.25em}
% Formatação do texto do título para a caixinha de princípio
{
  \small\sffamily\thmname{#1}\nobreakspace\thmnumber{\@ifnotempty{#1}{}\@upn{#2}}
  \thmnote{\nobreakspace\the\thm@notefont\sffamily\bfseries---\nobreakspace#3.}
}

% Define a forma de numerar os princípios
\theoremstyle{prinstyle}
\newtheorem{prinT}{Princípio}[chapter]

% Define a caixa para as princípios
\newmdenv[
  skipabove=5pt,
  skipbelow=5pt,
  rightline=true,
  leftline=true,
  topline=false,
  bottomline=false,
  linecolor=vOrange,
  backgroundcolor=vGray,
  innerleftmargin=5pt,
  innerrightmargin=5pt,
  innertopmargin=3pt,
  innerbottommargin=3pt,
  leftmargin=0cm,
  rightmargin=0cm,
  linewidth=3pt]{prinBox}

% Cria o ambiente para o princípios
\newenvironment{principio}{\begin{prinBox}\begin{prinT}}{\end{prinT}\end{prinBox}}

%---------------------------------------------------------------------------------------
%	Define a caixinha para o ambiente  de princípio
%---------------------------------------------------------------------------------------

% nome do estilo para usar nos princípio
\newtheoremstyle{lemastyle}
% Espaço acima
{0pt}
% Espaço abaixo
{0pt}
% Fonte dentro da caixa
{\normalfont}
% Indentação
{}
% Título
{\small\bf\sffamily}
% Pontuação pós título
{\;}
% Espaço depois do título
{0.25em}
% Formatação do texto do título para a caixinha de princípio
{
  \small\sffamily\thmname{#1}\nobreakspace\thmnumber{\@ifnotempty{#1}{}\@upn{#2}}
  \thmnote{\nobreakspace\the\thm@notefont\sffamily\bfseries---\nobreakspace#3.}
}

% Define a forma de numerar os princípios
\theoremstyle{lemastyle}
\newtheorem{lemaT}{Lema}[chapter]

% Define a caixa para as princípios
\newmdenv[
  skipabove=5pt,
  skipbelow=5pt,
  rightline=true,
  leftline=true,
  topline=true,
  bottomline=true,
  linecolor=Salmon,
  backgroundcolor=white,
  innerleftmargin=5pt,
  innerrightmargin=5pt,
  innertopmargin=3pt,
  innerbottommargin=3pt,
  leftmargin=0cm,
  rightmargin=0cm,
  linewidth=1pt]{lemaBox}

% Cria o ambiente para o princípios
\newenvironment{lema}{\begin{lemaBox}\begin{lemaT}}{\end{lemaT}\end{lemaBox}}


%---------------------------------------------------------------------------------------
%	Define a caixinha para o ambiente  de questões
%---------------------------------------------------------------------------------------

%\theoremstyle{testStyle}
\newtheorem{testeT}{Questão}[chapter]
%\newenvironment{questaos}{\begin{testeT}$({\color{cyan}\star})$}{\end{testeT}}
\newenvironment{questao}{\begin{testeT}}{\end{testeT}}

% Comando Necessário
\makeatother

