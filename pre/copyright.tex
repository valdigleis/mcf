%----------------------------------------------------------------------------------------
%	Páginas de Copyright
%----------------------------------------------------------------------------------------
\begingroup

\newpage
\thispagestyle{empty}

%\noindent Valdigleis S. Costa

%\noindent Professor Adjunto, Colegiado de Ciência da Computação

%\noindent Universidade Federal do Vale do São Francisco

%\noindent Salgueiro, PE\\

\noindent \textit{Copyright} \copyright\ 2019-{\the\year} Valdigleis S. Costa

\noindent Este texto  \textsc{NÃO}  possui qualquer tipo de vínculo editorial, e não possui fins lucrativos.

\noindent Página pessoal do autor \url{https://valdigleis.site}

~\vfill

\thispagestyle{empty}

\begin{figure*}[h]
	\centering
	\includegraphics[width=0.15\linewidth]{fig/license}
\end{figure*}
\noindent Este material é licenciado sob a Licença Atribuição-NãoComercial-CompartilhaIgual 3.0 Não Adaptada (CC BY-NC-SA 4.0).  Você pode obter uma copia da licença acessando a página: 
\begin{center}
	\url{https://creativecommons.org/licenses/by-nc-sa/4.0/legalcode.pt}
\end{center}
\noindent ou enviando uma carta para Creative Commons, 444 Castro Street, Suite 900, Mountain View, California, 94041, USA.

~\vfill

\hrule
\vspace*{1cm}

\noindent Este tomo foi escrito com base em uma coleção de notas de aulas do autor, o mesmo foi redigido usando um \textit{template} desenvolvido pelo próprio autor. Este texto foi escrito com o conjunto de macros {\LaTeX} ({\color{NordAurora1}em sua versão 2}) e compilado usado as ferramentas {\LuaLaTeX} e {Bib\TeX}, tais ferramentas fornecidas pelas distribuições {\TeX}Live e Mac{\TeX}, respectivamente  nos sistema operacionais \textit{Unix-like}: Gnu/Debian e no Mac OS X, para edição foram usados os \textit{softwares} livres  de edição textual NeoVim ({\color{NordAurora1}versão 0.10.1}), além disso, o sistema de controle de versão adotado é o Git ({\color{NordAurora1}versão 2.34.1}). \\ 

%\noindent \textit{Release} compilado em \today (\currenttime) -- \MinutesAfterMidnight\ minutos após a meia-noite. %
\noindent \textit{Release} compilado em \today\ (\MinutesAfterMidnight\ minutos após a meia-noite). %

\endgroup
