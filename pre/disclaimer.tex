%----------------------------------------------------------------------------------------
%	Página de Disclaimer
%----------------------------------------------------------------------------------------
\begingroup

\newpage \thispagestyle{empty} \
\newpage

\thispagestyle{empty}
\begin{center}
	{\normalfont\fontsize{20}{20}\sffamily\selectfont {\color{NordAurora5}\faLightbulb} \\ \textbf{\textit{Sobre este documento}}}\par
\end{center}

%\vspace{0.75cm}

Este documento vem sendo construído aos poucos ({\color{NordAurora1}em passos de tartaruga}), tendo como base diversas notas de aula (manuscritas a mão) que eu preparei  para ministrar cursos de graduação nos seguintes tópicos:


\begin{multicols}{2}
	\begin{fieldsList}
		\item Conjuntos, relações e funções;
		\item Lógica;
		\item Álgebra universal;
		\item Teoria dos códigos;
		\item Linguagem formais e autômatos;
		\item Computabilidade e decidibilidade;
		\item Análise de algoritmos;
		\item Grafos;
		\item Categorias;
		\item Teoria da Informação;
	\end{fieldsList}
\end{multicols}	

Uma vez que este documento ainda é um projeto em andamento e possivelmente sua escrita nunca será realmente concluída com total aprovação de seu autor, é claro que você poderá encontrar diversos erros, que com toda certeza você leitor irá me enviar e-mails\footnote{E-mail do autor: \url{valdigleis@gmail.com}} ou \textit{issues}\footnote{Páginas de \textit{issues}: \url{https://gitlab.com/valdigleis/mcf/-/issues}} com reports de tais erros, no caso de ser meu aluno também pode fazer apontamentos através da comunidade \textbf{extra-classe}\footnote{Acessível através do link \url{https://valdigleis.site/extraclasse}}. Para acessar a página da nova versão deste documento basta escanear o \textbf{QR \textit{code}} abaixo.

\

\begin{figure*}[h]
	\centering
	\includegraphics[width=0.2\linewidth]{fig/qrcode}
\end{figure*}

Para finalizar, a personagem que você irá encontrar de forma recorrente neste documento chama-se ALiCIA\footnote{ALiCIA é um acrônimo para Autômatos, Linguagens, Complexidade, Informação e Algoritmos.}, ela é uma criação do autor deste livro e todas as imagens da mesma são de propriedade do autor, não sendo permitido o usado das imagens por terceiros sem autorização assinada pelo autor deste documento.

\endgroup
\newpage
