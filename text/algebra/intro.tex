\chapter{Fundamentos}

\epigraph{``Não deve haver, entre os teoremas da matemática, um enunciado que, em certo sentido, não tenha sido provado de uma maneira algébrica.''}{David Hilbert.}

\section{Introdução}

%Na matemática, uma álgebra é um objeto estrutural, ou seja, é uma estrutura com partes bem definidas. 

Matemáticos em geral dizem simplesmente que, uma álgebra consiste em um ou mais conjuntos de objetos e uma ou mais operações sobre esses objetos \cite{lang2008}. Já para os cientistas da computação (e também alguns lógicos\footnote{O autor tem bons amigos lógicos com esse pensamento.}), a álgebra tem o significado mais mecânico, ela é, na verdade, uma máquina abstrata, onde os conjuntos de objetos, são os dados sobre os quais a máquina trabalha, e as operações são os programas que a máquina consegue executar.

Independente da visão adotada, estamos interessados nas propriedades das operações (programas) envolvidas, por exemplo, quais são suas aridades e quais leis ou axiomas elas satisfazem. Obviamente, como é um objeto matemático, as álgebras podem ser tipadas, assim, qualquer álgebra tem um tipo associado a ela. E será o tipo que especifica o número de operações (programas) que a álgebra possui, além de, fornecer informações sobre a aridade de cada operação (programa). 

A maioria dos estudantes de matemática em sua graduação estuda uma disciplina chamada álgebra abstrata, que é nada mais que o estudo de grupos e aneis, sendo estes, entretanto, apenas dois tipos de álgebras \cite{stanley1981}. Por outro lado, a álgebra não se resume a aneis e grupos, de fato a ágebra é uma área bem mais ampla, e o estudo das álgebras em sua totalidade, isto é, sem se prender a um tipo de álgebra particular é a área da matemática que hoje é chamada de álgebra universal \cite{klaus2001, benja-Logica}, que será o assunto estudo a partir da próxima seção deste documento.


\section{Definições Fundamentais}

Este texto começa então apresentando os conceito relacionados a sintaxe da álgebra sem se preocupar inicialmente com a semântica, para isso primeiro será apresentado o conceito de assinatura algébrica, ou simplesmente, $\Sigma$-assinatura (ou apenas assinatura).

\begin{definicao}[Assinatura]\label{def:SigmaAssinatura}
  Uma $\Sigma$-Assinatura, ou simplesmente, assinatura, é uma estrutura $\langle \Sigma, arid \rangle$, onde $\Sigma$ é um conjunto enumerável de símbolos e $arid : \Sigma \rightarrow \mathbb{N}$ é uma função total, tal que para cada $f \in \Sigma$ tem-se que $arid(f)$ representa a aridade\footnote{Aridade de um símbolo $f$ é a quantidade de argumentos (outros símbolos) que podem ser associados por vez ao símbolo $f$.} de $f$.
\end{definicao}

\begin{exemplo}\label{exe:SigmaAssinatura1}
  As estruturas abaixo são ambas assinaturas,
  \begin{itemize}
    \item[(a)] $\langle \{+, -, *, p \}, arid_1 \rangle$ com $arid_1(+) = arid_1(-) = arid_1(*) = arid_1(p) = 2$.
    \item[(b)] $\langle \{|, a, +\}, arid_2 \rangle$ com $arid_2(|) = 2,  arid_2(a) = 0$ e $arid_2(+) = 4$.
  \end{itemize}
  Por outro lado, a estrutura $\langle \{+, 0, 1, 2\}, arid_3 \rangle$ onde tem-se a definição 
  $$arid_3(0) = arid_3(1) = arid_3(2) = 0$$ 
  não é uma assinatura.
\end{exemplo}

Obviamente, ficar escrevendo a estrutura completa de uma assinatura, isto é, escrever $\langle \Sigma, arid \rangle$, é algo tedioso e muitas vezes desnecessário, assim a menos que seja de suma importância, sempre que for necessário falar de uma assinatura será escrito apenas o $\Sigma$ ficando a existência da função $arid$ subentendida.

\begin{nota}[Nomenclatura.]
  ALiCIA, além de, adorar matemática abstrata, também gosta muito de programação! Assim, para este texto ficar com um vocabulário próximo da programação, ela decidiu que deste ponto em diante, os símbolos em $\Sigma$ serão chamados de símbolos funcionais.
\end{nota}

A Definição \ref{def:SigmaAssinatura} apresenta os símbolos (os funcionais) que foram parte do alfabeto básico da linguagem que será descrita por uma $\Sigma$-álgebra, note também que, além dos funcionais as regras sintáticas já foram apresentadas de forma não detalhada, a seguir tais regras são apresentadas de forma explícita.

\

\

\begin{definicao}\label{def:SintaxeAlgebra}
  Seja $\Sigma$ uma assinatura, a sintaxe descrita por $\Sigma$ é formada pelas seguintes regras: 
  \begin{itemize}
    \item[i.] Todo $f \in \Sigma$ com $arid(f) = 0$ é tal que $f$ é uma palavra válida.
    \item[ii.] Para todo $f \in \Sigma$ se $arid(f) = n$, então a palavra $f(x_1, \cdots, x_n)$ é uma palavra válida, onde todo $x_i$ com $1 \leq i \leq n$ é uma palavra válida, um símbolo de uma variável, ou um símbolo de um conjunto não vazio dado.
  \end{itemize}
\end{definicao}

Antes de prosseguir é interessante falar sobre a escrita de palavra válidas usando funcionais de aridade $2$. Para estes funcionais existe duas formas muito utilizadas na literatura (ver \cite{klaus2001, stanley1981}) para escrever palavras válidas: 

\begin{enumerate}
  \item A forma que é apresentada no item (ii) da Definição \ref{def:SintaxeAlgebra} que é chamada de \textbf{escrita prefixa}, que nada mais é do que a escrita natural de funções que foi já apresentada no Capítulo \ref{cap:Functions}. 
  \item A forma \textbf{infixa}: na forma infixa um funcional $f$ de aridade $2$ seria usando entre os objetos que estão com ele relacionados, isto é, a palavra válida seria da forma $x_1 f x_2$\footnote{Muitas linguagens de programação como Haskell \cite{learnHaskell2011, beginningHaskell} e provadores de teoremas tais como Coq \cite{coq2013} permite usar tanto a notação prefixa como a notação infixa.}.
\end{enumerate}

Agora para continuar o estudo das $\Sigma$-álgebra será necessário que a seguinte concessão seja feita, para todo conjunto não vazio $A$ tem-se que $A^0 = \{\emptyset\}$, tendo feito essa concessão pode-se finalmente apresentar a definição a seguir.

\begin{definicao}[$\Sigma$-Álgebra]\label{def:SigmaAlgebra}
  Seja $\Sigma$ uma assinatura, uma $\Sigma$-álgebra é uma estrutura $\langle A, \Sigma_A \rangle$, onde:
  \begin{itemize}
    \item[(i)] $A$ um conjunto não vazio e 
    \item[(ii)] $\Sigma_A = \{f: A^n \rightarrow A  \mid n \in \mathbb{N}, f \in \Sigma, arid(f) = n\}$ é um conjunto finito e não vazio de funções totais\footnote{Aqui está sendo usado apenas a assinaturas das funções por critérios de simplificação de escrita, mas o leitor deve lembrar que uma função é uma estrutura bem mais rica (se necessário visite o capítulo \ref{cap:Functions}).}, chamado conjunto de interpretação de $\Sigma$.
  \end{itemize}
\end{definicao}

Com a Definição \ref{def:SigmaAssinatura} são apresentados símbolos funcionais, depois na Definição \ref{def:SintaxeAlgebra} é apresentado como usar tais símbolos para criar palavras válidas dentro da linguagem descrita pela $\Sigma$-álgebra, entretanto, não se tinha qualquer tipo de significado para os símbolos funcionais, ou seja, eles eram objetos puramente sintáticos. Na Definição \ref{def:SigmaAlgebra} usando um conjunto base (os dados) $A$ é dado significado aos símbolos funcionais através do conjunto $\Sigma_A$, eles (os funcionais) se tornam verdadeiramente objetos (funções) com um significado, ou seja, a $\Sigma$-álgebra de fato atribui sintaxe e semântica para uma linguagem, observe entretanto, que a semântica é definida pelo conjunto base e apenas por ele. 

Para entender melhor o que foi dito no parágrafo anterior, considere por exemplo, uma assinatura $\langle \{+\}, arid \rangle$ com $arid(+)  = 2$, note que, dado o conjunto dos números naturais $\mathbb{N}$ e o conjunto $Por$ das palavras da linguagem portuguesa pode-se definir duas $\{+\}$-álgebras, sendo elas: $\langle \mathbb{N}, \{+ :\mathbb{N}^2 \rightarrow \mathbb{N}\} \rangle$ e $\langle Por, \{+ :Por^2 \rightarrow Por\} \rangle$, note que a Definição \ref{def:SigmaAlgebra} impõem que o funcional $+$ seja uma função total em ambas as álgebras, porém, $+$ terá interpretações (significado) diferentes em cada álgebra, neste caso particular em $\mathbb{N}$ o funcional pode ser interpretado com a soma usual de números naturais, por outro lado, poderia ser interpretado como a concatenação\footnote{A concatenação de duas palavras $a_1\cdots a_m$ e $b_1\cdots b_n$ é a operação que produz a palavra $a_1\cdots a_mb_1\cdots b_n$ como resultado.} de palavras, o que importa que o leitor perceba é que o funcional $+$ será interpretado de forma diferente em cada álgebra isoladamente.

\begin{nota}[Um super poder da álgebra.]
  Como $\Sigma$-álgebras são linguagens\footnote{Máquinas na linguagem da ciência da computação.} isolado, pode-se ter duas álgebras distintas para o mesmo conjunto de funcionais usando o mesmo conjunto básico de dados para dar semântica aos funcionais, por exemplo, considere a assinatura $\langle \{+\}, arid \rangle$ com $arid(+)  = 2$ e o conjunto dos naturais $\mathbb{N}$, pode-se apresentar duas (ou até mais) $\Sigma$-álgebras $A = \langle \mathbb{N}, \{+ :\mathbb{N}^2 \rightarrow \mathbb{N}\} \rangle$ e $B = \langle \mathbb{N}, \{+ :\mathbb{N}^2 \rightarrow \mathbb{N}\} \rangle$, na álgebra $A$ a interpretação de $+$ pode sem problemas ser a soma padrão de naturais, já na álgebra $B$ o funcional $+$ poderia ser interpretado (ou seja, definido em $\Sigma_\mathbb{N}$) usando notação infixa, para todo $x, y \in \mathbb{N}$ da seguinte forma:
  \begin{eqnarray*}
    x + y = \left\{\begin{array}{ll}	0, & \hbox{se } x = y\\1,  & \hbox{senão}\end{array}\right.
  \end{eqnarray*}
  Assim a definição de $\Sigma$-álgebra, permite criar álgebras distintas usando a mesma $\Sigma$-assinatura e o mesmo conjunto base $X$, o que muda é apenas a interpretação dos funcionais, ou seja, o conjunto $\Sigma_X$ é o que irá difereir nas $\Sigma$-álgebras.
\end{nota}

\begin{exemplo}\label{exe:AlgebraUniversal1}
  A estrutura da forma $\langle \mathbb{Z}, \{S : \mathbb{Z}^3 \rightarrow \mathbb{Z}, P : \mathbb{Z} \rightarrow \mathbb{Z}, t:\{\emptyset\} \rightarrow \mathbb{Z} \} \rangle$ com as seguintes interpretações:
  \begin{itemize}
    \item $S(x, y, z) = 2x + z - y$.
    \item $P(x) = x - 1$.
    \item $t(\emptyset) = 14$.
  \end{itemize}
  Onde $+$ e $-$ são interpretados com a soma e a subtração usual dos inteiros, é um exemplo bem definido de $\Sigma$-Álgebra.
\end{exemplo}

Agora como tido em \cite{klaus2001}, dado uma $\Sigma$-ágebra $\langle A, \Sigma_A \rangle$, para todos os símbolos funcionais $f \in \Sigma$ com $arid(f) = 0$, sempre irá existir um elemento $a \in A$, tal que $f(\emptyset) = a$, nesse caso é comum escreve o funcional como $f_a$ ou simplesmente como $a$, em vez, do símbolo funcional $f$ explicitamente, em outras palavras, os funcionais de aridade $0$ representam valores constantes dentro de qualquer álgebra\footnote{Provalmente o leitor já teve um contato com essa ideia anteriormente, por exemplo, o símbolo $\pi$ pode ser facilmente interpretado como sendo um funcional de aridade $0$ que representa um valor real constante, cujo valor é aproximadamente $3.14$.}.

\begin{nota}[Rótulos para $\Sigma$-álgebras.]
  Da mesma forma que aparece no Capítulo \ref{cap:Sets}, onde rótulo são usados para representar conjuntos, as $\Sigma$-álgebras também podem receber um rótulo (apelido ou nome), isso é feito apresentando o rótulo seguindo do símbolo de igualdade e da estrutura da $\Sigma$-álgebra, por exemplo, se é necessário rotular a $\Sigma$-álgebra $\langle A, \Sigma_A \rangle$ pelo símbolo $\mathcal{A}$, basta mencionar que $\mathcal{A} = \langle A, \Sigma_A \rangle$, e depois disso sempre que esta álgebra for mencionada pode-se usar simplesmente o símbolo $\mathcal{A}$ para se referir a álgebra, em vez de, ficar escrevendo a estrutura $\langle A, \Sigma_A \rangle$.
\end{nota}

O próximo passo no estuado da álgebra universal é a entender a tarefa de classificar as $\Sigma$-álgebras em diferentes tipos (ou classes), e isto é exposto na definição a seguir.

\begin{definicao}[Tipo de uma $\Sigma$-álgebra]
  Seja $\mathcal{A} = \langle A, \Sigma_A \rangle$ uma $\Sigma$-álgebra, o tipo de $\mathcal{A}$ é uma família sequêncial de números naturais $\{x_i\}_{i \in I}$ com $I \subseteq \mathbb{N}$, tal que para todo $i, j \in I$ tem-se que se $i \leq j$, então $x_i \geq x_j$ com $arid(f) = x_i$ e $arid(g) = x_j$ sendo que $f, g 
  \in \Sigma$.
\end{definicao}

\begin{exemplo}\label{exe:TipoSigmaAlgebra1}
  A $\Sigma$-álgebra $\langle A, \Sigma_A \rangle$ onde $\Sigma = \{ *, \mid, e \}$ com $arid(*) = 1, arid(\mid) = 2$ e $arid(e) = 0$ é do tipo $\{2, 1, 0\}$
\end{exemplo}

\begin{exemplo}\label{exe:TipoSigmaAlgebra2}
  A $\Sigma$-álgebra $\langle \wp(\mathbb{N}), \Sigma_{\wp(\mathbb{N})} \rangle$ onde $\Sigma = \{ \cup, \cap, \vee, \wedge, \{0\}, \emptyset \}$ com $arid(\emptyset) = arid(\{0\}) = 0$ e $arid(x) = 2$ sendo que $x \in \{\cup, \cap, \vee, \wedge\}$ e $arid(e) = 0$ é do tipo $\{2, 2, 2, 2, 0, 0\}$.
\end{exemplo}

\begin{exemplo}\label{exe:TipoSigmaAlgebra3}
  A $\Sigma$-álgebra $\langle \{0, 1, 2, 3, 4\}, \Sigma_{\{0, 1, 2, 3, 4\}} \rangle$ onde $\Sigma = \{ max, p, 4, 3 \}$ com $arid(max) = 2, arid(p) = 4$, $arid(3) = 0$ e $arid(4) = 0$ é do tipo $\{4, 2, 0, 0\}$. Onde $max$ é o máximo usual sobre os reais, $p(a, b, c, d) = min(max(a, d), max(b, c))$ para todo $a, b, c, d \in \{0, 1, 2, 3, 4\}$ e $3$ e $4$ são constantes.
\end{exemplo}

Apesar de no Exemplos \ref{exe:TipoSigmaAlgebra3} anterior ter sido exposta um funcional de aridade 4 (a saber o funcional $p$), isso não acontece com muita frequência nas estruturas das $\Sigma$-álgebras, pois, como dito em \cite{stanley1981}, é comum (mas não obrigatório\footnote{O próprio \cite{stanley1981}, apresenta em \textsection \ Exemplo 8, um funcional com aridade maior que 2.}) os funcionais terem aridade menor ou igual a 2.

Agora uma vez que, dado uma $\Sigma$-álgebra $\langle A, \Sigma_A \rangle$, na definição de assinatura o conjunto $\Sigma$ não é necessariamente ordenado, e devido a isso, a interpretação $\Sigma_A$ também não será necessariamente um conjunto ordenado. Entretanto, como dito em \cite{carmo2013}, sempre que $\Sigma$ é um conjunto finito, o mesmo é visto como sendo um conjunto ordenado, onde a ordem dos símbolos obdece extamente a tipagem da $\Sigma$-álgebra associada, isto é, os elementos elementos de $\Sigma$ são ordenado de forma decrescente com respeito a imagem de $arid$. E neste cenário como mostrado em \cite{carmo2013, klaus2001, stanley1981}, se $\Sigma = \{f_1, \cdots, f_n\}$ com $aird(f_i) \geq arid(f_{i+1})$ para todo $1 \leq i \leq n-1$, então para todo conjunto $A$ a $\Sigma$-álgebra é escrita na forma (do açúcar sintático) $\langle A, f_1, \cdots, f_n \rangle$. 

\begin{nota}
  Dado uma $\Sigma$-álgebra $\mathcal{A} = \langle A, \Sigma_A \rangle$, é dita que $\mathcal{A}$ é uma álgebra finita quando $A$ é finito, e no caso particular de $\#A = 1$, é dito que $\mathcal{A}$ é uma álgebra trivial.
\end{nota}

\section{subálgebras}\label{sec:SubAlgebras}

O próximo aspecto de importante relevância dentro do estuda da teoria das álgebras universais é o conceito de subálgebra, formalizado a seguir. 

\begin{definicao}[subálgebra]\label{def:SubAlgebras}
  Seja $\mathcal{B} = \{B, \Sigma_B\}$ uma $\Sigma$-álgebra de tipo $\rho$, a estrutura $\mathcal{A} = \{A, \Sigma_A\}$ é uma sub-ágebra de $\mathcal{B}$, denotado por $\mathcal{A} \preccurlyeq  \mathcal{B}$, se e somente se, ela satisfaz as seguintes condições:
  \begin{itemize}
    \item[i.] $\mathcal{A}$ também é do tipo $\rho$.
    \item[ii.] $A \subseteq B$ e
    \item[iii.] Para todo $f \in \Sigma$ tem-se a relação\footnote{Aqui usa-se as notações $f_\mathcal{A}$ e $f_\mathcal{B}$ para denotar a interpretação do funcional $f$ na $\Sigma$-álgebra $\mathcal{A}$ e na $\Sigma$-álgebra $\mathcal{B}$ respectivamente.} $ima(f_\mathcal{A}) \subseteq ima(f_\mathcal{B})$.
  \end{itemize}
\end{definicao}

O resultado a seguir apresenta uma forma de checar se uma $\Sigma$-álgebra é subálgebra de outra $\Sigma$-álgebra, tal resultado é em geral nomeado como criterio de sub-ágebra.

\begin{teorema}[Criterio de sub-ágebra]
  A estrutura $\mathcal{A} = \{A, \Sigma_A\}$ é uma sub-ágebra de $\mathcal{B} = \{B, \Sigma_B\}$ se, e somente se, para todo $n \in \mathbb{N}$ e todo $X \subseteq A^n$, para todo  $f \in \Sigma$ sempre que $arid(f) = n$ tem-se que $\overrightarrow{f_\mathcal{B}}[X] \subseteq A$.
\end{teorema}

\begin{proof}
  $(\Rightarrow)$ Suponha que $\mathcal{A} = \{A, \Sigma_A\}$ é uma sub-ágebra de $\mathcal{B} = \{B, \Sigma_B\}$, assim pela Definição \ref{def:SubAlgebras} tem-se que:
  \begin{itemize}
    \item[i.] $\mathcal{A}$ também é do tipo $\rho$.
    \item[ii.] $A \subseteq B$ e
    \item[iii.] Para todo $f \in \Sigma$ tem-se a relação\footnote{Aqui usa-se as notações $f_\mathcal{A}$ e $f_\mathcal{B}$ para denotar a interpretação do funcional $f$ na $\Sigma$-álgebra $\mathcal{A}$ e na $\Sigma$-álgebra $\mathcal{B}$ respectivamente.} $ima(f_\mathcal{A}) \subseteq ima(f_\mathcal{B})$.
  \end{itemize}
  Agora, uma vez que, $\mathcal{A}$ e $\mathcal{B}$ tem o mesmo tipo, tem-se obviamente que para todo $f$, sua interpretação terá a mesma quantidade de elementos como entrada da função, ou seja, se $arid(f) = n$ segue que $(x_1, \cdots, x_n) \in A^n$  e $(x'_1, \cdots, x'_n) \in B^n$ são as entradas de $f_\mathcal{A}$ e $f_\mathcal{B}$ respectivamente, agora por $(ii)$ é claro que $(x_1, \cdots, x_n) \in B^n$, finalmente, uma vez que, por $(iii)$ tem-se  que $ima(f_\mathcal{A}) \subseteq ima(f_\mathcal{B})$ e as funções $f_\mathcal{A}$ e $f_\mathcal{B}$ são totais tais que $ima(f_\mathcal{A}) \subseteq A$ e $ima(f_\mathcal{B}) \subseteq B$, é claro que, $f_\mathcal{B}(x_1, \cdots, x_n) \in A$ e desde que $(x_1, \cdots, x_n) \in A^n$, pode-se concluir que para todo $n \in \mathbb{N}$ e todo $X \subseteq A^n$, para todo  $f \in \Sigma$ sempre que $arid(f) = n$ tem-se que $\overrightarrow{f_\mathcal{B}}[X] \subseteq A$.

  $(\Leftarrow)$ Fica como exercício ao leitor.
\end{proof}

Imediatamente desde resultado pode-se concluir os dois importantes resultados da teoria das $\Sigma$-álgebras expostos a seguir.

\begin{corolario}
  Sejam $\mathcal{A} = \{A, \Sigma_A\}, \mathcal{B} = \{B, \Sigma_B\}$ e $\mathcal{C} = \{C, \Sigma_C\}$ três $\Sigma$-álgebras de tipo $\rho$, tem-se que:
  \begin{itemize}
    \item[(a)] Se $\mathcal{A} \preccurlyeq  \mathcal{B}$ e $\mathcal{B} \preccurlyeq  \mathcal{C}$, então $\mathcal{A} \preccurlyeq  \mathcal{C}$.
    \item[(b)] Se $A \subseteq B \subseteq C$ e $\mathcal{A} \preccurlyeq  \mathcal{C}$ e $\mathcal{B} \preccurlyeq  \mathcal{C}$, então $\mathcal{A} \preccurlyeq  \mathcal{C}$.
  \end{itemize}
\end{corolario}

\begin{proof}
  A prova fica como exercício de fixação ao leitor.
\end{proof}

\begin{corolario}
  A interseção de uma família $\{\mathcal{A}_i\}_{i \in I}$ de subálgebras de uma $\Sigma$-álgebra $\mathcal{B}$ de tipo $\rho$ é também uma subálgebra de $\mathcal{B}$.
\end{corolario}

\begin{proof}
  A prova fica como exercício de fixação ao leitor.
\end{proof}

Agora, para poder prosseguir é necessário considerar uma $\Sigma$-álgebra $\mathcal{B}$ de tipo $\rho$ e um subconjunto não vazio $X$ de seu conjunto base $B$. Convém questionar, \textbf{será existe uma ``menor'' subálgebra de $\mathcal{B}$ cujo conjunto base contém o conjunto $X$}? A resposta para essa pergunta é sim! Para entender, entretanto tal resposta, considere a seguinte definição.

\begin{definicao}
  Dado uma $\Sigma$-álgebra $\langle B, \Sigma_B \rangle$, para todo $X \subseteq B$ tem-se que:
  \begin{eqnarray*}
    Sg(X) = \bigcap \{ A \mid X \subseteq A, \langle A, \Sigma_A \rangle \preccurlyeq  \langle B, \Sigma_B \rangle \}
  \end{eqnarray*}
\end{definicao}